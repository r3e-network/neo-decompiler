\documentclass[11pt]{article}
\usepackage[a4paper,margin=1in]{geometry}
\usepackage{hyperref}
\usepackage{longtable}
\usepackage{array}
\usepackage{enumitem}
\usepackage{booktabs}

\title{Neo Decompiler Specification}
\author{Neo Decompiler Project}
\date{\today}

\begin{document}
\maketitle

\section{Scope and Purpose}
This specification defines the behaviour, inputs, outputs, and guarantees of the
\texttt{neo-decompiler} tool and library. It applies to the following deliverables:
\begin{itemize}[nosep]
    \item The Rust library \texttt{neo\_decompiler}.
    \item The CLI binary \texttt{neo-decompiler}.
    \item The generated metadata tables for opcodes, syscalls, and native contracts.
    \item The human-readable views (pseudocode, high-level, C\# skeleton) and
    JSON reports emitted by the CLI.
\end{itemize}

Implementations shall conform to the functional requirements in
Sections~\ref{sec:inputs}--\ref{sec:limits}. Non-normative notes are marked as
``(informative)''.

\section{Inputs and Dependencies}\label{sec:inputs}
\begin{itemize}[nosep]
    \item \textbf{NEF input:} A valid Neo N3 NEF file. The parser shall reject
    files with incorrect magic, checksum, or malformed fields.
    \item \textbf{Manifest input (optional):} A JSON manifest matching the NEF
    script. When present, manifest data shall be used to annotate outputs and
    drive entry-point naming.
    \item \textbf{Generated metadata:} The tool shall include the complete
    opcode table (from \texttt{src/opcodes\_generated.rs}), syscall table
    (\texttt{src/syscalls\_generated.rs}), and native contract table
    (\texttt{src/native\_contracts\_generated.rs}) without requiring network
    access at runtime.
\end{itemize}

\section{Functional Requirements}
\subsection{Disassembly}
\begin{itemize}[nosep]
    \item The disassembler shall decode all opcodes defined in the generated
    metadata and preserve byte offsets.
    \item Unsupported or unknown opcodes shall be represented with mnemonic
    placeholders but may not cause failure of the entire disassembly.
    \item Operands shall be parsed according to the operand encoding declared by
    each opcode.
\end{itemize}

\subsection{Higher-Level Views}
\begin{itemize}[nosep]
    \item The pseudocode view shall mirror the instruction stream with offsets
    and resolved syscall/native labels.
    \item The high-level view shall lift stack operations into readable
    statements and reconstruct structured control flow including \texttt{if},
    \texttt{else}, \texttt{while}, \texttt{do--while}, counted loops rewritten
    as \texttt{for}, and intra-loop \texttt{break}/\texttt{continue}.
    \item When a manifest is provided, the entry function name and parameters
    in the high-level view shall match the ABI method at the corresponding
    offset. Argument slot names shall align with manifest parameter names.
    \item The C\# view shall build a compilable skeleton that mirrors the
    manifest ABI, includes manifest extras (\texttt{Author}, \texttt{Email}) as
    \texttt{[ManifestExtra]} attributes, and preserves the same control-flow
    lifting as the high-level view.
\end{itemize}

\subsection{CLI Behaviour}
\begin{itemize}[nosep]
    \item The CLI shall expose subcommands \texttt{info}, \texttt{disasm},
    \texttt{decompile}, \texttt{tokens}, and \texttt{catalog}.
    \item Each subcommand shall support text and JSON formats; the
    \texttt{decompile} subcommand shall additionally support pseudocode,
    high-level, and C\# outputs.
    \item When a manifest is not provided explicitly, the CLI shall attempt to
    locate a sibling \texttt{.manifest.json} file.
\end{itemize}

\section{Outputs}\label{sec:outputs}
\subsection{Text Outputs}
\begin{itemize}[nosep]
    \item \textbf{info:} Header fields, script hashes, checksum, manifest
    summary, and method tokens with native annotations.
    \item \textbf{disasm:} Instruction listing with offsets and operands.
    \item \textbf{decompile:} One of pseudocode, high-level, C\#, or all views
    depending on \texttt{--format}.
    \item \textbf{tokens:} Method tokens with call flags and native hints.
    \item \textbf{catalog:} Syscalls, native contracts, or opcodes in tabular
    form.
\end{itemize}

\subsection{JSON Outputs}
JSON schemas reside in \texttt{docs/schema}. Table~\ref{tab:json-fields} lists
the top-level fields for the \texttt{decompile} report.

\begin{longtable}{@{}p{0.22\linewidth}p{0.72\linewidth}@{}}
\caption{\texttt{decompile} JSON fields}\label{tab:json-fields}\\
\toprule
Field & Description \\
\midrule
file & Path to the NEF file consumed.\\
manifest\_path & Path to the manifest, if supplied or discovered.\\
script\_hash\_le / script\_hash\_be & Script hashes in little/big endian hex.\\
high\_level & Lifted textual representation.\\
pseudocode & Raw instruction listing with offsets.\\
csharp & C\# skeleton derived from high-level view and manifest.\\
instructions & Array of decoded instructions with operands.\\
method\_tokens & Tokens with hashes, call flags, and native hints.\\
manifest & Manifest summary (methods, events, permissions, trusts).\\
warnings & Array of human-readable warnings (e.g., unknown native methods).\\
\bottomrule
\end{longtable}

\section{Compliance Matrix}\label{sec:compliance}
Table~\ref{tab:features} enumerates mandatory capabilities.

\begin{table}[h]
\centering
\begin{tabular}{@{}p{0.3\linewidth}p{0.65\linewidth}@{}}
\toprule
Capability & Requirement \\
\midrule
Opcode coverage & All opcodes from \texttt{src/opcodes\_generated.rs} decoded.\\
Syscall resolution & Syscalls mapped via \texttt{src/syscalls\_generated.rs} with names and call flags.\\
Native contracts & Method tokens annotated using \texttt{src/native\_contracts\_generated.rs}.\\
Control flow lifting & \texttt{if/else}, \texttt{while}, \texttt{do--while}, \texttt{for}, \texttt{break}/\texttt{continue}.\\
Manifest integration & Entry signature, parameter names, extras (Author/Email) applied to outputs.\\
Output formats & Text and JSON for all commands; pseudocode, high-level, C\# for \texttt{decompile}.\\
\bottomrule
\end{tabular}
\caption{Mandatory capabilities}\label{tab:features}
\end{table}

\section{Figure (Informative)}
Figure~\ref{fig:pipeline} illustrates the processing pipeline.

\begin{figure}[h]
\centering
\begin{tabular}{|c|c|c|}
\hline
NEF \& optional manifest & Disassembler \& metadata tables & Views \\
\hline
Input validation & Opcode/syscall/native lookup & Pseudocode / High-level / C\# / JSON \\
\hline
\end{tabular}
\caption{Processing pipeline (informative)}\label{fig:pipeline}
\end{figure}

\section{Limitations}\label{sec:limits}
\begin{itemize}[nosep]
    \item Type inference beyond manifest-provided metadata is out-of-scope.
    \item Complex control-flow graph recovery (e.g., switch lowering, try/finally
    reconstruction) is not guaranteed.
    \item Unknown opcodes beyond the generated table are rendered as
    placeholders but not lifted.
\end{itemize}

\section{Versioning}
This document SHALL be updated when any of the following occur:
\begin{itemize}[nosep]
    \item The opcode, syscall, or native contract tables change.
    \item New output formats or structured control-flow constructs are added.
    \item JSON schemas are modified.
\end{itemize}

\end{document}
